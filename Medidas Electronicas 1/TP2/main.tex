\documentclass[12pt, letterpaper]{article} 
\usepackage[utf8]{inputenc}
\usepackage[spanish]{babel} %pone en español textos automaticos
\usepackage{fancyhdr} % Se agrega el paquete fancyhdr
\usepackage{graphicx} % Se agrega el paquete graphicx
\usepackage[a4paper, margin=3cm]{geometry} %marjenes y hoja A4


%pies de pagina
\pagestyle{fancy} % Se especifica el estilo de página como fancy
\fancyhf{} % Se limpian los encabezados y pies de página predefinidos
\rfoot{Página \thepage\ de \pageref{LastPage}} %texto derecha
\lfoot{Medidas Electrónicas I} %texto izquierda
%encabezado de pagina
\chead{UNIVERSIDAD TECNOLÓGICA NACIONAL - FRC} % centro
\fancyhead[R]{\includegraphics[height=1cm]{imagenes/UTN_logo.jpg}}


\begin{document}

%Caratula
    \begin{titlepage}
        \centering %texto centrado
        {\includegraphics[width=0.2\textwidth]{imagenes/UTN_logo.jpg}\par}
        {\bfseries\LARGE Universidad Tecnológica Nacional \par}
        {\scshape\Large Facultad Regional Córdoba\par}
        \vspace{0.5cm}
        {\scshape\Huge Trabajo Práctico de laboratorio Nro. 2  \par}%titulo
        \raggedright %texto a la izquierda
        \vspace{0.5cm}
        {\Large Materia: Medidas Electrónicas 1 \par}%Materia
        \vspace{0.5cm}
        {\Large Curso: 4R1 \par}
        \vspace{0.5cm}
        {\Large Edificio: \par}%edificios
            \begin{itemize}
                \item{\Large Ingeniero Soro [Aula 606] \par}
                \item{\Large Laboratorio de electrónica \par}
            \end{itemize}
        \vspace{0.5cm}
        {\Large Profesores: \par} %profes
            \begin{itemize}
                \item{\Large [Teórico] Ing, Carlos Augusto Centeno \par}
                \item{\Large [Teórico] Ing, Luis Alberto Guanuco \par}
                \item{\Large [Práctico] Ing, Martin Alejandro Salamero \par}
            \end{itemize}
        \vspace{0.5cm}
        {\Large Autores: \par} %autores
            \begin{itemize}
                \item{\Large Pappano Meinardi, Joaquín - Leg.86730\par}
                \item{\Large Monteros Vigueras, Juan Manuel - Leg.86334\par}
                \item{\Large Romero Diaz, Agustín - Leg.86821\par}
            \end{itemize}
        \vspace{0.5cm}
        {\Large Fecha: {\today} \par}%pone fecha de hoy
    \end{titlepage}

%Indice
\newpage
\tableofcontents
\newpage

%cuerpo documento
\section{Introducción}

Aquí va la introducción del documento.

\subsection{Objetivo}
Determinar el valor de la resistencia en situaciones específicas. Dar el resultado de la
medición acompañado del grado de incertidumbre. 

\subsection{Materiales e Instrumental}
\begin{itemize}
    \item Multimetro digital con su correspondiente manual de especificaciones.
    \item Circuito generador de corriente constante.
    \item Probeta a ensayar (Tramo de cable o alambre de longitud conocida).
    \item Telurimetro UNI-T Modelo UT522 (Con su juego de cables, electrodos y accesorios)
\end{itemize}

\subsubsection{prueba}

\section{Desarrollo primer parte}

Aquí se describe el desarrollo de el estudio.

\section{Desarrollo segunda parte}

parte secundaria


\label{LastPage}

\end{document}