\documentclass[12pt, letterpaper]{article} 
\usepackage[utf8]{inputenc}
\usepackage{setspace}
\usepackage[spanish]{babel} %pone en español textos automaticos
\usepackage{fancyhdr} % Se agrega el paquete fancyhdr
\usepackage{graphicx} % Se agrega el paquete graphicx
\usepackage[a4paper, margin=3cm]{geometry} %marjenes y hoja A4


%pies de pagina.
\pagestyle{fancy} % Se especifica el estilo de página como fancy
\fancyhf{} % Se limpian los encabezados y pies de página predefinidos
\rfoot{Página \thepage\ de \pageref{LastPage}} %texto derecha
\lfoot{Electronica Aplicada II} %texto izquierda
%encabezado de pagina
\chead{UNIVERSIDAD TECNOLÓGICA NACIONAL - FRC} % centro
\fancyhead[R]{\includegraphics[height=1cm]{Imagenes/UTN_logo.jpg}}


\begin{document}

%Caratula
\begin{titlepage}
	\centering %texto centrado
	{\includegraphics[width=0.2\textwidth]{Imagenes/UTN_logo.jpg}\par}
	{\bfseries\LARGE Universidad Tecnológica Nacional \par}
	{\scshape\Large Facultad Regional Córdoba\par}
	\vspace{1cm}
	{\scshape\Huge Amplificador Realimentado V-S \par}%titulo
	\raggedright %texto a la izquierda
	\vspace{1cm}
	{\Large Materia: Electronica Aplicada II \par}%Materia
	\vspace{0.5cm}
	{\Large Curso: 4R1 \par}
	\vspace{0.5cm}
	{\Large Edificio: \par}%edificios
	\begin{itemize}
		\item{\Large Ingeniero Soro [Aula 606] \par}
		\item{\Large Laboratorio de electrónica \par}
	\end{itemize}
	\vspace{0.5cm}
	{\Large Profesores: \par} %profes
	\begin{itemize}
		\item{\Large [Teórico] Ing, Carlos Celdran \par}
		\item{\Large [Teórico] Ing, Carlos Enrique Olmos \par}
		\item{\Large [Práctico] Ing, Federico Luis José Linares \par}
	\end{itemize}
	\vspace{0.5cm}
	{\Large Autores: \par} %autores
	\begin{itemize}
		\item{\Large Pappano Meinardi, Joaquín - Leg.86730\par}
		\item{\Large Monteros Vigueras, Juan Manuel - Leg.86334\par}
		\item{\Large Romero Diaz, Agustín - Leg.86821\par}
	\end{itemize}
	\vspace{0.5cm}
	{\Large Fecha: {\today} \par}%pone fecha de hoy
\end{titlepage}

%Indice
\newpage
\tableofcontents
\newpage

%cuerpo documento
\section{Introducción}

La retroalimentación negativa es una técnica ampliamente utilizada en el diseño de circuitos electrónicos para mejorar la estabilidad, la linealidad y la precisión de los amplificadores.
El amplificador de retroalimentación negativa V-S es un ejemplo común de esta técnica, que utiliza una red beta y un amplificador multietapas para reducir la ganancia y mejorar la linealidad del circuito.
En este informe, se presentará una descripción detallada del diseño y funcionamiento del amplificador de retroalimentación negativa V-S, así como su análisis teórico y experimental.
Además, se discutirán algunas de las aplicaciones prácticas de este tipo de amplificador en la electrónica analógica y su ancho de banda.

\section{Cálculos Teóricos}
Se realizaron los cálculos de $A_v$, $Z_o$, $Z_i$, $A_{vf}$, $Z_{of}$, $Z_{if}$
\singlespacing
Estamos frente a un amplificador el cual tiene unas función de transferencia $H(s)=\frac{V_o}{V_i}
$ por lo tanto el circuito equivalente de este lo podemos ver en la figura \ref{fig:2.1}
\begin{figure}[h]
	\centering
	\includegraphics[width=0.75\textwidth]{Imagenes/Screenshot_57.png}
	\caption{Circuito equivalente amplificador realimentado V-S}
	\label{fig:2.1}
\end{figure}
\singlespacing
definiremos los parámetros de este circuito respecto al esquemático dado
\singlespacing
$R_B=R_{b1}//R_{b2} \rightarrow 332K8$\hspace{1cm} $hie1=hie2 \rightarrow$\hspace{1cm} $R_e1 \rightarrow 18K$
\singlespacing
$Rf \rightarrow 1K6$ \hspace{1cm} $R_B'=R_{b3}//R_{b4} \rightarrow 19K7$ \hspace{1cm} $R_{c1} \rightarrow18K$
\singlespacing
$hfe \rightarrow $ \hspace{1cm} $R_{c2} \rightarrow 2K$
\singlespacing
Las ecuaciones que podemos obtener del análisis del circuito son
\singlespacing
$i_{b1}=V_i\frac{1}{hie_1+[(R_{e1}//R_f)(hfe + 1)]} \rightarrow \frac{i_{b1}}{v_i}=\frac{1}{hie_1+[(R_{e1}//R_f)(hfe + 1)]}$
\singlespacing
$ i_{b2}=ib1(-hfe\frac{R_{e1}//R_b'//hie_2}{hie_2}) \rightarrow \frac{i_{b2}}{i_{b1}}=-hfe\frac{R_{e1}//R_b'//hie_2}{hie_2}$
\singlespacing
$V_o=i_{b2}(-hfe[R_{e2}//(R_f+R_{e2})]) \rightarrow \frac{V_o}{i_{b2}}=-hfe[R_{e2}//(R_f+R_{e2}]$
\subsection{Calculo Av}
$A_v=\frac{V_o}{V_i}=\frac{v_o}{i_{b2}}\frac{i_{b2}}{i_{b1}}\frac{i_{b1}}{v_i}$
\subsection{Calculo Red beta y calculo Avf}
\begin{figure}[h]
	\centering
	\includegraphics[width=0.75\textwidth]{Imagenes/Screenshot_58.png}
	\caption{Red Beta  V-S}
	\label{fig:2.2.1}
\end{figure}
\singlespacing
$v_o=i(R_f+R_{e1})$ \hspace{1cm} $vf=iR_{e1}$
\singlespacing
$\beta = \frac{v_f}{v_o}=\frac{R_{e1}}{R_{e1}+R_f}$
\singlespacing
$D=|1+A_v\beta|$ \hspace{1cm} $A_{vf}=\frac{A_v}{D}$
\subsection{Cálculos de Zi,Zo,Zif,Zof}
Lazo abierto:
$Z_i=R_b//{hie_1+[(R_{e1}//R_f)(hfe+1)]}$
\singlespacing
$Z_o=R_{c2}//(R_{e1}+R_f)$
\singlespacing
Lazo cerrado :
$Z_{if}=Z_iD$
\singlespacing
$Z_{of}=\frac{Z_o}{D}$

\section{Simulaciones}




\end{document}