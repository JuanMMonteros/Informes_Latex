\documentclass[12pt, letterpaper]{article} 
\usepackage[utf8]{inputenc}
\usepackage{setspace}
\usepackage[spanish]{babel} %pone en español textos automaticos
\usepackage{fancyhdr} % Se agrega el paquete fancyhdr
\usepackage{graphicx} % Se agrega el paquete graphicx
\usepackage{float} %para agregar imagenes con H
\usepackage{pdfpages} %agregar pdfs para datasheets y eso
\usepackage[a4paper, margin=3cm]{geometry} %marjenes y hoja A4


%pies de pagina.
\pagestyle{fancy} % Se especifica el estilo de página como fancy
\fancyhf{} % Se limpian los encabezados y pies de página predefinidos
\rfoot{Página \thepage\ de \pageref{LastPage}} %texto derecha
\lfoot{Electrónica Aplicada II} %texto izquierda
%encabezado de pagina
\chead{UNIVERSIDAD TECNOLÓGICA NACIONAL - FRC} % centro
\fancyhead[R]{\includegraphics[height=1cm]{Imagenes/UTN_logo.jpg}}


\begin{document}

%Caratula
\begin{titlepage}
	\centering %texto centrado
	{\includegraphics[width=0.2\textwidth]{Imagenes/UTN_logo.jpg}\par}
	{\bfseries\LARGE Universidad Tecnológica Nacional \par}
	{\scshape\Large Facultad Regional Córdoba\par}
	\vspace{1cm}
	{\scshape\Huge Amplificador Realimentado V-S \par}%titulo
	\raggedright %texto a la izquierda
	\vspace{1cm}
	{\Large Materia: Electrónica Aplicada II \par}%Materia
	\vspace{0.5cm}
	{\Large Curso: 4R1 \par}
	\vspace{0.5cm}
	{\Large Edificio: \par}%edificios
	\begin{itemize}
		\item{\Large Ingeniero Soro [Aula 606] \par}
		\item{\Large Laboratorio de electrónica \par}
	\end{itemize}
	\vspace{0.5cm}
	{\Large Profesores: \par} %profes
	\begin{itemize}
		\item{\Large [Teórico] Ing, Carlos Celdran \par}
		\item{\Large [Teórico] Ing, Carlos Enrique Olmos \par}
		\item{\Large [Práctico] Ing, Federico Luis José Linares \par}
	\end{itemize}
	\vspace{0.5cm}
	{\Large Autores: \par} %autores
	\begin{itemize}
		\item{\Large Pappano Meinardi, Joaquín - Leg.86730\par}
		\item{\Large Monteros Vigueras, Juan Manuel - Leg.86334\par}
		\item{\Large Romero Diaz, Agustín - Leg.86821\par}
	\end{itemize}
	\vspace{0.5cm}
	{\Large Fecha: {\today} \par}%pone fecha de hoy
\end{titlepage}

%Indice
\newpage
\tableofcontents
\newpage

%cuerpo documento
\section{Introducción}

La retroalimentación negativa es una técnica ampliamente utilizada en el diseño de circuitos electrónicos para mejorar la estabilidad, la linealidad y la precisión de los amplificadores.
El amplificador de retroalimentación negativa V-S, amplificador de tensión con muestra de tensión en serie, es un ejemplo común de esta técnica, que utiliza una red beta y circuito multietapas para reducir la ganancia pero mejorar la linealidad del circuito y su estabilidad.
En este informe, se presentará una descripción detallada del diseño y funcionamiento del amplificador de retroalimentación negativa V-S (figura \ref{fig:circuito}), así como su análisis teórico y experimental.
Además, se discutirán su ancho de banda y algunas de las aplicaciones prácticas de este tipo de amplificador en la electrónica analógica.


\begin{figure}[H]
    \centering
    \includegraphics[width=1\textwidth]{Imagenes/Circuito.png}
    \caption{Esquema circuito amplificador V-S}
    \label{fig:circuito}
\end{figure}
\singlespacing
El circuito implementado para este trabajo puede verse en la figura \ref{fig:circuito_placa}, para este se utilizaron transistores NPN BC548, véase la datasheet en el anexo figura \ref{fig:datasheet_bc548}.

\begin{figure}[H]
    \centering
    \includegraphics[width=0.5\textwidth]{Imagenes/Placa.jpg}
    \caption{placa con circuito implementado}
    \label{fig:circuito_placa}
\end{figure}

\section{Cálculos Teóricos}
Se realizaron los cálculos de $A_v$, $Z_o$, $Z_i$, $A_{vf}$, $Z_{of}$, $Z_{if}$
\singlespacing
Estamos frente a un amplificador el cual tiene unas función de transferencia $H(s)=\frac{V_o}{V_i}
$ por lo tanto el circuito equivalente de este es como lo podemos ver en la figura \ref{fig:cir_equivalente_señal}
\begin{figure}[H]
	\centering
	\includegraphics[width=1\textwidth]{Imagenes/circeq.png}
	\caption{Circuito equivalente amplificador realimentado V-S}
	\label{fig:cir_equivalente_señal}
\end{figure}
\singlespacing
definiremos los parámetros de este circuito respecto al esquemático dado
\singlespacing
$R_B=R_{b1}//R_{b2} \rightarrow 332K8\Omega$\hspace{1cm} 
\singlespacing
$R_{e3} \rightarrow 240\Omega$\hspace{1cm} $R_{L} \rightarrow 470\Omega$ \hspace{1cm} $R_{f} \rightarrow 1k6\Omega$
\singlespacing
$Rf \rightarrow 1K6\Omega$ \hspace{1cm} $R_B'=R_{b3}//R_{b4} \rightarrow 19K7\Omega$ \hspace{1cm} $R_{c1} \rightarrow18K\Omega$
\singlespacing
$hfe1 \rightarrow 359$ \hspace{1cm} $hfe2 \rightarrow 339$ $R_{c2} \hspace{1cm} \rightarrow 2K\Omega$
\singlespacing
Para $hie$ se deben encontrar las corriente ICQ1 e ICQ2 Para eso analizamos la polarización de los transistores. vease en figura \ref{fig:pol_Q1} y \ref{fig:pol_Q2}
\begin{figure}[H]
	\centering
	\includegraphics[height=0.75\textwidth]{Imagenes/polarizacionQ1.png}
	\caption{Polarización Q1}
	\label{fig:pol_Q1}
\end{figure}
$Vcc-VR_{b1}-VR_{b2}=0 \rightarrow ICQ1=hfe1\frac{Vcc}{R_{b1}+R_{b2}}=5,73mA$
\singlespacing
$hie1=hfe1\frac{25mV}{ICQ1}=1568\Omega$
\begin{figure}[H]
	\centering
	\includegraphics[width=0.75\textwidth]{Imagenes/PolarizaQ2.png}
	\caption{Polarización Q2}
	\label{fig:pol_Q2}
\end{figure}
$Vcc-VR_{b3}-VR_{b4}=0 \rightarrow ICQ2=hfe2\frac{Vcc}{R_{b3}+R_{b4}}=55,65mA$
\singlespacing
$hie2=hfe2\frac{25mV}{ICQ2}=152\Omega$
\singlespacing

Las ecuaciones que podemos obtener del análisis del circuito son
\singlespacing
$i_{b1}=V_i\frac{1}{hie_1+[(R_{e1}//R_f)(hfe + 1)]} \rightarrow \frac{i_{b1}}{v_i}=\frac{1}{hie_1+[(R_{e1}//R_f)(hfe + 1)]}=22,36*10^{-6}$
\singlespacing
$ i_{b2}=ib1(-hfe\frac{R_{e1}//R_b'//(hie_2+R_{e3}(hfe_2+1))}{(hie_2+R_{e3}(hfe_2+1))}{hie_2}) \rightarrow \frac{i_{b2}}{i_{b1}}=-hfe\frac{R_{e1}//R_b'//(hie_2+R_{e3}(hfe_2+1))}{(hie_2+R_{e3}(hfe_2+1))}{hie_2}=35$
\singlespacing
$V_o=i_{b2}(-hfe[R_{e2}//(R_f+R_{e2})]) \rightarrow \frac{V_o}{i_{b2}}=-hfe[R_{e2}//(R_f+R_{e2}]=105749$
\subsection{Calculo Av}
$A_v=\frac{V_o}{V_i}=\frac{v_o}{i_{b2}}\frac{i_{b2}}{i_{b1}}\frac{i_{b1}}{v_i}=82,87$
\subsection{Calculo Red beta y calculo Avf}
\begin{figure}[H]
	\centering
	\includegraphics[width=0.75\textwidth]{Imagenes/Screenshot_58.png}
	\caption{Red Beta  V-S}
	\label{fig:Red_beta}
\end{figure}
\singlespacing
$v_o=i(R_f+R_{e1})$ \hspace{1cm} $vf=iR_{e1}$
\singlespacing
$\beta = \frac{v_f}{v_o}=\frac{R_{e1}}{R_{e1}+R_f}=0,075$
\singlespacing
$D=|1+A_v\beta|=7,1618$ \hspace{1cm} $A_{vf}=\frac{A_v}{D}=11,57$
\subsection{Cálculos de Zi,Zo,Zif,Zof}
Lazo abierto:
$Z_i=R_b//{hie_1+[(R_{e1}//R_f)(hfe+1)]}=39523\Omega$
\singlespacing
$Z_o=R_{c2}//(R_{e1}+R_f)=927,61\Omega$
\singlespacing
Lazo cerrado :
$Z_{if}=Z_i*D=283K\Omega$
\singlespacing
$Z_{of}=\frac{Z_o}{D}=130\Omega$
\section{Simulaciones}
Para la simulación se utilizo el software Ltspace en el mismo se modelo el amplificador realimentado, se selecciono un valor de $Vin=60mV_{pp}$ y una $f=1,5KHz$ mismos valores que se utilizaron en las pruebas de laboratorio.
El circuito en el software es el vito en la figura \ref{fig:circuito}.
\subsection{Lazo Abierto}
y se pudo observar la $Vout=4,8V_{pp}$  (figura \ref{fig:sim_Vo})
\begin{figure}[H]
	\centering
	\includegraphics[width=0.75\textwidth]{Imagenes/Av.png}
	\caption{Vout Lazo Abierto}
	\label{fig:sim_Vo}
\end{figure}
Se calculo la ganancia a lazo abierto $H(s)=\frac{V_o}{V_i} = 80$
\singlespacing
Se hizo un barrido en frecuencia para ver las variación de la ganancia, através de un diagrama de bode (figura \ref{fig:bode_lazo_abierto})
\begin{figure}[H]
	\centering
	\includegraphics[width=0.75\textwidth]{Imagenes/lazoAbierto.png}
	\caption{AV en función de la frecuencia}
	\label{fig:bode_lazo_abierto}
\end{figure}
Se puede observar que el ancho de banda va desde los 10Khz hasta los 2MHz
\subsection{Lazo Cerrado}
Al cerrar el lazo se volvió a medir la $Vout=700mV$ (figura \ref{fig:vof})
\begin{figure}[H]
	\centering
	\includegraphics[width=0.75\textwidth]{Imagenes/Avf.png}
	\caption{Vout Lazo cerrado}
	\label{fig:vof}
\end{figure}
Se calculo la ganancia a lazo abierto $H(s)=\frac{V_o}{V_i} = 11,66$
Se hizo un barrido en frecuencia para ver las variación de la ganancia, através de un diagrama de bode 
\begin{figure}[H]
	\centering
	\includegraphics[width=0.75\textwidth]{Imagenes/lazoCerrado.png}
	\caption{AVf en función de la frecuencia}
	\label{fig:bode_lazo_cerrado}
\end{figure}
Se puede observar que el ancho de banda va desde los 10Khz hasta los 10MHz (figura \ref{fig:bode_lazo_cerrado})





\section{Procedimiento}
\subsection{Consignas Solicitadas}
\begin{itemize}
	\item  Implementar un amplificador realimentado
    \item  Realizar las mediciones $A_v, Z_o,Z_i, A_{vf},Z_{of},Z_{if} $
    \item  Realizar la medición de Desensibilidad provocando una variación máxima de Av del 45 \% , para lo cual se deberá modificar algún componente del amplificador (usaremos Re3).
    \item Obtener y graficar la curva de respuesta en frecuencia de la ganancia circuito a lazo abierto y lazo cerrado del circuito implementado. Para esto se utilizará los generadores provistos por el laboratorio centra
    \item Haciendo uso del simulador verificar los valores obtenidos del punto anterior (ganancia y respuesta en frecuencia).
\end{itemize}
Para poder realizar las mediciones solicitadas , se realizo el pcb del circuito correspondiente y se montaron los componentes en el mismo de manera superficial.
Una vez hecho esto se alimento el circuito con 22V C.C, en paralelo con $R_L$ se conecto un osciloscopio digital para poder medir la salida y en la entrada un generador de ondas $V_{in}= 60 V_{pp}$ como en el esquema de figura \ref{fig:coneccion_lab}
\begin{figure}[H]
	\centering
	\includegraphics[width=0.75\textwidth]{Imagenes/conexion.png}
	\caption{Conexiones amplificador realimentado V-S}
	\label{fig:coneccion_lab}
\end{figure}
\subsection{Mediciones}
Se comenzó midiendo la salida a lazo abierto $V_o=4V_{pp}$, se procedió a conectar la realimentación a la salida y se obtuvo la $V_o=700mV$ a lazo cerrado
\singlespacing
$A_{v1}=\frac{V_o}{V_{i}}=\frac{4V}{0.06V}=66.66 \therefore A_{vf1}=\frac{V_o}{V_{i}}=\frac{700mV}{60mV}=11.66$
\singlespacing
Para reducir la ganancia a lazo abierto un 45 \% se remplazo $R_{e3}$ por un tripo de $1K\Omega$ y se lo vario hasta que la $V_o=2,2V$ en lazo abierto esto se logro con el tripo en $500\omega$, se conecto la realimentación en se midió $V_o=550mv$
\singlespacing
Con estos datos se volvió a calcular $A_{v2}=\frac{V_o}{V_{i}}=\frac{2,2V}{0.06V}=36.66 \therefore A_{vf2}=\frac{V_o}{V_{i}}=\frac{550mV}{60mV}=9.66$
\singlespacing
$D=\frac{\Delta \% A_v}{\Delta \% A_{vf}}=\frac{45\%}{21,44\%}=2,09$
\singlespacing
Una vez calculada la desensibilidad se conecto en la entrada un tripo de $100k\Omega$ con se muestra en la figura\ref{fig:med_Zi}.
\singlespacing
\begin{figure}[H]
	\centering
	\includegraphics[width=0.75\textwidth]{Imagenes/zi.png}
	\caption{Circuito para medición Zi}
	\label{fig:med_Zi}
\end{figure}
con el amplificador a lazo abierto se aumento la $V_i$ hasta llegar al punto MES(Máxima excursión simétrica) $V_o=4,8V$, se conecto el tripo y se lo vario hasta llegar a una $V_o=2,4V$, se saco el tripo y se lo medio con un ohmetro.
\singlespacing
$Z_i=40,26k\Omega$
\singlespacing

Para $Z_o$ se realizo la conexión como se indica en la figura\ref{fig:med_Zo}.

\singlespacing
\begin{figure}[H]
	\centering
	\includegraphics[width=0.75\textwidth]{Imagenes/zo.png}

	\caption{Circuito para medición Zo}
	\label{fig:med_Zo}
\end{figure}
Primero con el amplificador a lazo abierto se elimino la resistencia de carga y se inyecto una señal de entrada asta alcanzar el punto MES . Echo esto se se inserto un tripo de $10k\Omega$ en el lugar de la carga y se lo vario hasta que $V_{out}$ disminuyo en un 50\% su valor , se saco el tripo y se lo medio con un ohmetro.
\singlespacing
$Z_o=2k\Omega$
\singlespacing


\section{Respuesta en frecuencia}
Se midió la respuesta en frecuencias a laso abierto y cerrado del circuito (figura \ref{fig:circuito}). En el cuadro \ref{tab:tab_resp_frec} se observan los valores obtenidos en el laboratorio.
\begin{table}[H]
    \centering
    \caption{Respuesta en frecuencia medida en laboratorio}
    \begin{tabular}{|r|r|r|r|r|r}
        
    \hline \multicolumn{5}{|c|}{Variación de la ganancia a lazo } &  \\
    
    \hline  \multicolumn{1}{|l|}{Frecuencia(Hz)} & \multicolumn{1}{l|}{Vo-pp(mv)} & \multicolumn{1}{l|}{Vof-pp(mv)} & \multicolumn{1}{l|}{Av} & \multicolumn{1}{l|}{Avf} &  \\
    
    \cmidrule    10    & 2000  & 200   & 33,3333333 & 3,33333333 &  \\
    
    \cmidrule   50    & 4240  & 500   & 70,6666667 & 8,33333333 &  \\
    
    \cmidrule    100   & 4000  & 720   & 66,6666667 & 12    &  \\
    
    \cmidrule    300   & 4000  & 760   & 66,6666667 & 12,6666667 &  \\
    
    \cmidrule    500   & 4800  & 688   & 80    & 11,4666667 &  \\
    
    \cmidrule    1000  & 4644  & 688   & 77,4  & 11,4666667 &  \\
    
    \cmidrule    5000  & 5000  & 720   & 83,3333333 & 12    &  \\
    
    \cmidrule   8000  & 4760  & 720   & 79,3333333 & 12    &  \\
    
    \cmidrule   10000 & 4720  & 672   & 78,6666667 & 11,2  &  \\
    
    \cmidrule    50000 & 4800  & 680   & 80    & 11,3333333 &  \\
    
    \cmidrule   100000 & 4800  & 705   & 80    & 11,75 &  \\
    
    \cmidrule   200000 & 4600  & 720   & 76,6666667 & 12    &  \\
    
    \cmidrule   500000 & 3600  & 680   & 60    & 11,3333333 &  \\
    
    \cmidrule    800000 & 2720  & 680   & 45,3333333 & 11,3333333 &  \\
    
    \cmidrule    1000000 & 2400  & 680   & 40    & 11,3333333 &  \\
    \cmidrule   1500000 & 1560  & 720   & 26    & 12    &  \\
    \cmidrule    2000000 & 1120  & 720   & 18,6666667 & 12    &  \\
    \cmidrule    
    \end{tabular}
    \label{tab:tab_resp_frec}
\end{table}

\section{Concluciones}
Con este practico pudimos ver los efectos que causa la realimentación negativa en los amplificadores, mejorando las caracterısticas de los mismos, y a través de las practicas de laboratorio y la realización de los cálculos pudimos demostrarlo analıtica y prácticamente.
\singlespacing
se observo que las ventajas de la realimentación negativas como el aumento de ancho de banda y la gran estabilidad que le brinda al circuito,se midiendo la desensibilidad tanto en forma practica como teórico, teniendo que variar hasta en un 45\% la tencion de salida a lazo abierto para poder observar cambios en la tencion de salida a lazo abierto. Esto nos trae grandes beneficios ya que los transistores suelen variar muchos sus parámetros (hfe,hie,etc) entre uno y otro , con la temperatura y con la frecuencia 
\singlespacing
Otra de las ventajas que se observaron fue como mejoro su impedancia de entrada($Z_{if}$) la cual se multiplica por la desensibilidad aumentando la misma para no sacarle corriente a la fuente de señal.La impedancia de salida($Z_{of}$) se vio dividida por la desensibilidad para que la potencia valla hacia la carga.
\singlespacing
El uso de la realimentación esta directamente relacionado con el sacrificio de gran parte de la ganancia para mejorar la estabilidad del amplificador y sus impedancias tanto a la entra como a la salida, sin embargo esto puede solucionarse agregando una etapa amplificadora extra
%no mover esto del final
\newpage
\section{Anexo}
\begin{figure}[H]
    \centering
    \includepdf[pages={1}, scale=0.60]{Imagenes/BC548.pdf}
    \caption{Datasheet bc548}
    \label{fig:datasheet_bc548}
\end{figure}
\newpage
\begin{figure}[H]
    \centering
    \includepdf[pages={2}, scale=0.60]{Imagenes/BC548.pdf}
    \caption{Datasheet bc548}
\end{figure}


\label{LastPage}
\end{document}